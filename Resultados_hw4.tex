\documentclass[12pt]{article}
\usepackage[utf8]{inputenc}
\usepackage[english]{babel}
\usepackage{graphicx}
\usepackage{tabularx}
\usepackage[absolute]{textpos} 
\usepackage{multirow}
\usepackage{float}
\usepackage{hyperref}
\author{Juan Pablo Carvajal Parra}
\date{19 de Noviembre de 2018}
\title{Tarea 4 Metodos Computacionales}


\begin{document}

\section{Ecuaciones diferenciales ordinarias}

En este primer punto se calcula usando el metodo de Runge-Kutta de cuarto orden la trayectoria de un proyectil en 2 dimensiones  

\begin{figure}[h!]
    \centering
    \includegraphics[scale=0.3]{45.jpg}
    \caption{Trayectoria del proyectil en el plano x-y, cuando se lanza con un angulo de 45 grados}
    \label{fig1}
\end{figure}

En el primer caso donde el angulo era de 45 grados se puede observar con claridad el efecto de la resistencia del aire sobre la trayectoria del proeyctil, en la primera mitad del recorrido es una trayectoria muy cercana a la parabolica pero en la segunda mitad el proyectil pierde velocidad y empieza a caer rapidamente.

\newpage
\begin{figure}[h!]
    \centering
    \includegraphics[scale=0.3]{Angulos.jpg}
    \caption{Trayectoria del proyectil en el plano x-y, cuando se lanza con varios angulos}
    \label{fig2}
\end{figure}

Cuando se lanza con distintos angulos es claro que el proyectil alcanza su mayor distancia en el eje x con el angulo de 20 grados, en cada trayectoria es notorio el efecto de la resistencia del aire sobre el proyectil

\newpage
\section{Ecuaciones diferenciales parciales}

En este punto se nos pide que analizemos la evolucion de la temperatura en una seccion cuadrada de una piedra de calcita en contacto con una varilla a 100 grados centigrados. Mis unidades para este punto fueron gramos,kelvin, centimetros y joules. El sistema era una seccion cuadrada de 50 cm de calcita a 283.15 Kelvin en cuyo centro estaba una varilla de 10 cm de diametro a 373.15 Kelvin.

\begin{figure}[h!]
    \centering
    \includegraphics[scale=0.3]{Momento_inicial.jpg}
    \caption{Condiciones iniciales del problema}
    \label{fig3}
\end{figure}

\newpage
\begin{figure}[h!]
    \centering
    \includegraphics[scale=0.3]{1_fijas.jpg}
    \caption{Primera evolucion del sistema con fronteras fijas}
    \label{fig4}
\end{figure}


\begin{figure}[h!]
    \centering
    \includegraphics[scale=0.3]{2_fijas.jpg}
    \caption{Segunda evolucion del sistema con fronteras fijas}
    \label{fig5}
\end{figure}

\newpage
\begin{figure}[h!]
    \centering
    \includegraphics[scale=0.3]{Fronteras_fijas.jpg}
    \caption{Sistema en equilibrio con fronteras fijas}
    \label{fig12}
\end{figure}

En el primer caso del problema, con condiciones de frontera a 283.15 Kelvin, se observa que la temperatura de la calcita crece cerca a la varilla y se propaga a los bordes hasta alcanzar una condicion de equilibrio despues de 0.05 segundos, como se espera la temperatura promedio del sistema no supera los 373.15 Kelvin


\newpage
\begin{figure}[h!]
    \centering
    \includegraphics[scale=0.3]{1_abiertas.jpg}
    \caption{Primera evolucion del sistema con fronteras abiertas}
    \label{fig6}
\end{figure}


\begin{figure}[h!]
    \centering
    \includegraphics[scale=0.3]{2_abiertas.jpg}
    \caption{Segunda evolucion del sistema con fronteras abiertas}
    \label{fig7}
\end{figure}

\newpage
\begin{figure}[h!]
    \centering
    \includegraphics[scale=0.3]{Fronteras_abiertas.jpg}
    \caption{Sistema en equilibrio con fronteras abiertas}
    \label{fig8}
\end{figure}

En el caso donde las condiciones de frontera son abiertas el sistema sigue incrementando en temperatura hasta que este alcanza la temperatura mas alta que haya, la cual es 373.15 Kelvin, se oberva que las esquinas de la seccion no crecen tan rapido como las demas partes y esto es debido a que los bordes tienen mas cercana la varilla.


\newpage
\begin{figure}[h!]
    \centering
    \includegraphics[scale=0.3]{1_periodicas.jpg}
    \caption{Primera evolucion del sistema con fronteras periodicas}
    \label{fig9}
\end{figure}


\begin{figure}[h!]
    \centering
    \includegraphics[scale=0.3]{2_periodicas.jpg}
    \caption{Primera evolucion del sistema con fronteras periodicas}
    \label{fig10}
\end{figure}

\begin{figure}[h!]
    \centering
    \includegraphics[scale=0.3]{Fronteras_periodicas.jpg}
    \caption{Sistema en equilibrio con fronteras periodicas}
    \label{fig11}
\end{figure}
\newpage

Con condiciones de frontera periodicas el sistema evoluciona muy parecido al caso anterior, por un corto periodo de tiempo el sistema con fronteras periodicas aumenta mas rapido su temperatura a comparacion con el otro, pero entre 0.2 y 0.3 segundos ya no hay diferencia notable y ambos alcanzan su temperatura de equilibrio al mismo tiempo. 



\newpage
\begin{figure}[h!]
    \centering
    \includegraphics[scale=0.3]{Temperatura_promedio.jpg}
    \caption{Evolucion de la temperatura promedio del sistema a traves del tiempo}
    \label{fig13}
\end{figure}








\end{document}
