\documentclass[12pt]{article}
\usepackage[utf8]{inputenc}
\usepackage[spanish]{babel}
\usepackage{graphicx}
\usepackage{tabularx}
\usepackage[absolute]{textpos} % Para poner una imagen en posiciones arbitrarias
\usepackage{multirow}
\usepackage{float}
\usepackage{hyperref}
\author{Juan Pablo Carvajal Parra}
\date{19 de Noviembre de 2018}
\title{Tarea 4 Metodos Computacionales}
%\decimalpoint

\begin{document}

\section{Ecuaionces diferenciales ordinarias}

Aqui va la descripcion del primer punto

\begin{figue}
\includegraphics[scale=0.3]{45.jpg} 
\end{figure}

\begin{figue}
\includegraphics[scale=0.3]{Angulos.jpg} 
\end{figure}




\section{Ecuaciones diferenciales parciales}

Aqui va la descripcion del segundo punto



\end{document}

