\documentclass[12pt]{article}
\usepackage[utf8]{inputenc}
\usepackage[spanish]{babel}
\usepackage{graphicx}
\usepackage{tabularx}
\usepackage[absolute]{textpos} 
\usepackage{multirow}
\usepackage{float}
\usepackage{hyperref}
\author{Juan Pablo Carvajal Parra}
\date{19 de Noviembre de 2018}
\title{Tarea 4 Metodos Computacionales}


\begin{document}

\section{Ecuaciones diferenciales ordinarias}

En este primer punto se calcula usando el metodo de Runge-Kutta de cuarto orden la trayectoria de un proyectil en 2 dimensiones  

\begin{figure}[h!]
    \centering
    \includegraphics[scale=0.3]{45.jpg}
    \caption{Trayectoria del proyectil en el plano x-y, cuando se lanza con un angulo de 45 grados}
    \label{fig1}
\end{figure}

En el primer caso donde el angulo era de 45 grados se puede observar con claridad el efecto de la resistencia del aire sobre la trayectoria del proeyctil, en la primera mitad del recorrido es una trayectoria muy cercana a la parabolica pero en la segunda mitad el proyectil pierde velocidad y empieza a caer rapidamente.

\newpage
\begin{figure}[h!]
    \centering
    \includegraphics[scale=0.3]{Angulos.jpg}
    \caption{Trayectoria del proyectil en el plano x-y, cuando se lanza con varios angulos}
    \label{fig2}
\end{figure}

Cuando se lanza con distintos angulos es claro que el proyectil alcanza su mayor distancia en el eje x con el angulo de 20 grados


\newpage
\section{Ecuaciones diferenciales parciales}

En este punto se nos pide que analizemos la evolucion de una seccion cuadrada de una calcita en contacto con una varilla a 100 grados centigrados 


Aqui va la descripcion del segundo punto


\begin{figure}[h!]
    \centering
    \includegraphics[scale=0.3]{Momento_inicial.jpg}
    \caption{Condiciones iniciales del problema}
    \label{fig3}
\end{figure}


\begin{figure}[h!]
    \centering
    \includegraphics[scale=0.3]{1_fijas.jpg}
    \caption{Primera evolucion del sistema con fronteras fijas}
    \label{fig4}
\end{figure}


\begin{figure}[h!]
    \centering
    \includegraphics[scale=0.3]{2_fijas.jpg}
    \caption{Segunda evolucion del sistema con fronteras fijas}
    \label{fig5}
\end{figure}


\begin{figure}[h!]
    \centering
    \includegraphics[scale=0.3]{Fronteras_fijas.jpg}
    \caption{Sistema en equilibrio con fronteras fijas}
    \label{fig12}
\end{figure}



\begin{figure}[h!]
    \centering
    \includegraphics[scale=0.3]{1_abiertas.jpg}
    \caption{Primera evolucion del sistema con fronteras abiertas}
    \label{fig6}
\end{figure}


\begin{figure}[h!]
    \centering
    \includegraphics[scale=0.3]{2_abiertas.jpg}
    \caption{Segunda evolucion del sistema con fronteras abiertas}
    \label{fig7}
\end{figure}


\begin{figure}[h!]
    \centering
    \includegraphics[scale=0.3]{Fronteras_abiertas.jpg}
    \caption{Sistema en equilibrio con fronteras abiertas}
    \label{fig8}
\end{figure}


\begin{figure}[h!]
    \centering
    \includegraphics[scale=0.3]{1_periodicas.jpg}
    \caption{Primera evolucion del sistema con fronteras periodicas}
    \label{fig9}
\end{figure}


\begin{figure}[h!]
    \centering
    \includegraphics[scale=0.3]{2_periodicas.jpg}
    \caption{Primera evolucion del sistema con fronteras periodicas}
    \label{fig10}
\end{figure}

\begin{figure}[h!]
    \centering
    \includegraphics[scale=0.3]{Fronteras_periodicas.jpg}
    \caption{Sistema en equilibrio con fronteras periodicas}
    \label{fig11}
\end{figure}


\begin{figure}[h!]
    \centering
    \includegraphics[scale=0.3]{Temperatura_promedio.jpg}
    \caption{Evolucion de la temperatura promedio del sistema a traves del tiempo}
    \label{fig13}
\end{figure}








\end{document}
