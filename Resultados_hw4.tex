\documentclass[12pt]{article}
\usepackage[utf8]{inputenc}
\usepackage[spanish]{babel}
\usepackage{graphicx}
\usepackage{tabularx}
\usepackage[absolute]{textpos} 
\usepackage{multirow}
\usepackage{float}
\usepackage{hyperref}
\author{Juan Pablo Carvajal Parra}
\date{19 de Noviembre de 2018}
\title{Tarea 4 Metodos Computacionales}


\begin{document}

\section{Ecuaciones diferenciales ordinarias}

Aqui va la descripcion del primer punto

\begin{figure}[h!]
    \centering
    \includegraphics[scale=0.3]{45.jpg}
    \caption{Trayectoria del proyectil en el plano x-y}
    \label{fig1}
\end{figure}

\begin{figure}[h!]
    \centering
    \includegraphics[scale=0.3]{Angulos.jpg}
    \caption{Trayectoria del proyectil en el plano x-y}
    \label{fig2}
\end{figure}
















\section{Ecuaciones diferenciales parciales}

En este punto se nos pide que analizemos la evolucion de una seccion cuadrada de una calcita en contacto con una varilla a 100 grados centigrados 


Aqui va la descripcion del segundo punto


\begin{figure}[h!]
    \centering
    \includegraphics[scale=0.3]{Momento_inicial.jpg}
    \caption{Condiciones iniciales del problema}
    \label{fig3}
\end{figure}


\begin{figure}[h!]
    \centering
    \includegraphics[scale=0.3]{1_fijas.jpg}
    \caption{Primera evolucion del sistema con fronteras fijas}
    \label{fig4}
\end{figure}


\begin{figure}[h!]
    \centering
    \includegraphics[scale=0.3]{2_fijas.jpg}
    \caption{Segunda evolucion del sistema con fronteras fijas}
    \label{fig5}
\end{figure}


\begin{figure}[h!]
    \centering
    \includegraphics[scale=0.3]{Fronteras_fijas.jpg}
    \caption{Sistema en equilibrio con fronteras fijas}
    \label{fig12}
\end{figure}



\begin{figure}[h!]
    \centering
    \includegraphics[scale=0.3]{1_abiertas.jpg}
    \caption{Primera evolucion del sistema con fronteras abiertas}
    \label{fig6}
\end{figure}


\begin{figure}[h!]
    \centering
    \includegraphics[scale=0.3]{2_abiertas.jpg}
    \caption{Segunda evolucion del sistema con fronteras abiertas}
    \label{fig7}
\end{figure}


\begin{figure}[h!]
    \centering
    \includegraphics[scale=0.3]{Fronteras_abiertas.jpg}
    \caption{Sistema en equilibrio con fronteras abiertas}
    \label{fig8}
\end{figure}


\begin{figure}[h!]
    \centering
    \includegraphics[scale=0.3]{1_periodicas.jpg}
    \caption{Primera evolucion del sistema con fronteras periodicas}
    \label{fig9}
\end{figure}


\begin{figure}[h!]
    \centering
    \includegraphics[scale=0.3]{2_periodicas.jpg}
    \caption{Primera evolucion del sistema con fronteras periodicas}
    \label{fig10}
\end{figure}

\begin{figure}[h!]
    \centering
    \includegraphics[scale=0.3]{Fronteras_periodicas.jpg}
    \caption{Sistema en equilibrio con fronteras periodicas}
    \label{fig11}
\end{figure}


\begin{figure}[h!]
    \centering
    \includegraphics[scale=0.3]{Temperatura_promedio.jpg}
    \caption{Evolucion de la temperatura promedio del sistema a traves del tiempo}
    \label{fig13}
\end{figure}








\end{document}
